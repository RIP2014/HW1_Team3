
% Default to the notebook output style

    


% Inherit from the specified cell style.




    
\documentclass{article}

    
    
    \usepackage{graphicx} % Used to insert images
    \usepackage{adjustbox} % Used to constrain images to a maximum size 
    \usepackage{color} % Allow colors to be defined
    \usepackage{enumerate} % Needed for markdown enumerations to work
    \usepackage{geometry} % Used to adjust the document margins
    \usepackage{amsmath} % Equations
    \usepackage{amssymb} % Equations
    \usepackage[mathletters]{ucs} % Extended unicode (utf-8) support
    \usepackage[utf8x]{inputenc} % Allow utf-8 characters in the tex document
    \usepackage{fancyvrb} % verbatim replacement that allows latex
    \usepackage{grffile} % extends the file name processing of package graphics 
                         % to support a larger range 
    % The hyperref package gives us a pdf with properly built
    % internal navigation ('pdf bookmarks' for the table of contents,
    % internal cross-reference links, web links for URLs, etc.)
    \usepackage{hyperref}
    \usepackage{longtable} % longtable support required by pandoc >1.10
    \usepackage{booktabs}  % table support for pandoc > 1.12.2
    

    
    
    \definecolor{orange}{cmyk}{0,0.4,0.8,0.2}
    \definecolor{darkorange}{rgb}{.71,0.21,0.01}
    \definecolor{darkgreen}{rgb}{.12,.54,.11}
    \definecolor{myteal}{rgb}{.26, .44, .56}
    \definecolor{gray}{gray}{0.45}
    \definecolor{lightgray}{gray}{.95}
    \definecolor{mediumgray}{gray}{.8}
    \definecolor{inputbackground}{rgb}{.95, .95, .85}
    \definecolor{outputbackground}{rgb}{.95, .95, .95}
    \definecolor{traceback}{rgb}{1, .95, .95}
    % ansi colors
    \definecolor{red}{rgb}{.6,0,0}
    \definecolor{green}{rgb}{0,.65,0}
    \definecolor{brown}{rgb}{0.6,0.6,0}
    \definecolor{blue}{rgb}{0,.145,.698}
    \definecolor{purple}{rgb}{.698,.145,.698}
    \definecolor{cyan}{rgb}{0,.698,.698}
    \definecolor{lightgray}{gray}{0.5}
    
    % bright ansi colors
    \definecolor{darkgray}{gray}{0.25}
    \definecolor{lightred}{rgb}{1.0,0.39,0.28}
    \definecolor{lightgreen}{rgb}{0.48,0.99,0.0}
    \definecolor{lightblue}{rgb}{0.53,0.81,0.92}
    \definecolor{lightpurple}{rgb}{0.87,0.63,0.87}
    \definecolor{lightcyan}{rgb}{0.5,1.0,0.83}
    
    % commands and environments needed by pandoc snippets
    % extracted from the output of `pandoc -s`
    \DefineVerbatimEnvironment{Highlighting}{Verbatim}{commandchars=\\\{\}}
    % Add ',fontsize=\small' for more characters per line
    \newenvironment{Shaded}{}{}
    \newcommand{\KeywordTok}[1]{\textcolor[rgb]{0.00,0.44,0.13}{\textbf{{#1}}}}
    \newcommand{\DataTypeTok}[1]{\textcolor[rgb]{0.56,0.13,0.00}{{#1}}}
    \newcommand{\DecValTok}[1]{\textcolor[rgb]{0.25,0.63,0.44}{{#1}}}
    \newcommand{\BaseNTok}[1]{\textcolor[rgb]{0.25,0.63,0.44}{{#1}}}
    \newcommand{\FloatTok}[1]{\textcolor[rgb]{0.25,0.63,0.44}{{#1}}}
    \newcommand{\CharTok}[1]{\textcolor[rgb]{0.25,0.44,0.63}{{#1}}}
    \newcommand{\StringTok}[1]{\textcolor[rgb]{0.25,0.44,0.63}{{#1}}}
    \newcommand{\CommentTok}[1]{\textcolor[rgb]{0.38,0.63,0.69}{\textit{{#1}}}}
    \newcommand{\OtherTok}[1]{\textcolor[rgb]{0.00,0.44,0.13}{{#1}}}
    \newcommand{\AlertTok}[1]{\textcolor[rgb]{1.00,0.00,0.00}{\textbf{{#1}}}}
    \newcommand{\FunctionTok}[1]{\textcolor[rgb]{0.02,0.16,0.49}{{#1}}}
    \newcommand{\RegionMarkerTok}[1]{{#1}}
    \newcommand{\ErrorTok}[1]{\textcolor[rgb]{1.00,0.00,0.00}{\textbf{{#1}}}}
    \newcommand{\NormalTok}[1]{{#1}}
    
    % Define a nice break command that doesn't care if a line doesn't already
    % exist.
    \def\br{\hspace*{\fill} \\* }
    % Math Jax compatability definitions
    \def\gt{>}
    \def\lt{<}
    % Document parameters
    \title{HW3}
    
    
    

    % Pygments definitions
    
\makeatletter
\def\PY@reset{\let\PY@it=\relax \let\PY@bf=\relax%
    \let\PY@ul=\relax \let\PY@tc=\relax%
    \let\PY@bc=\relax \let\PY@ff=\relax}
\def\PY@tok#1{\csname PY@tok@#1\endcsname}
\def\PY@toks#1+{\ifx\relax#1\empty\else%
    \PY@tok{#1}\expandafter\PY@toks\fi}
\def\PY@do#1{\PY@bc{\PY@tc{\PY@ul{%
    \PY@it{\PY@bf{\PY@ff{#1}}}}}}}
\def\PY#1#2{\PY@reset\PY@toks#1+\relax+\PY@do{#2}}

\expandafter\def\csname PY@tok@c\endcsname{\let\PY@it=\textit\def\PY@tc##1{\textcolor[rgb]{0.25,0.50,0.50}{##1}}}
\expandafter\def\csname PY@tok@s\endcsname{\def\PY@tc##1{\textcolor[rgb]{0.73,0.13,0.13}{##1}}}
\expandafter\def\csname PY@tok@s1\endcsname{\def\PY@tc##1{\textcolor[rgb]{0.73,0.13,0.13}{##1}}}
\expandafter\def\csname PY@tok@sh\endcsname{\def\PY@tc##1{\textcolor[rgb]{0.73,0.13,0.13}{##1}}}
\expandafter\def\csname PY@tok@kt\endcsname{\def\PY@tc##1{\textcolor[rgb]{0.69,0.00,0.25}{##1}}}
\expandafter\def\csname PY@tok@gi\endcsname{\def\PY@tc##1{\textcolor[rgb]{0.00,0.63,0.00}{##1}}}
\expandafter\def\csname PY@tok@vg\endcsname{\def\PY@tc##1{\textcolor[rgb]{0.10,0.09,0.49}{##1}}}
\expandafter\def\csname PY@tok@mf\endcsname{\def\PY@tc##1{\textcolor[rgb]{0.40,0.40,0.40}{##1}}}
\expandafter\def\csname PY@tok@sr\endcsname{\def\PY@tc##1{\textcolor[rgb]{0.73,0.40,0.53}{##1}}}
\expandafter\def\csname PY@tok@si\endcsname{\let\PY@bf=\textbf\def\PY@tc##1{\textcolor[rgb]{0.73,0.40,0.53}{##1}}}
\expandafter\def\csname PY@tok@c1\endcsname{\let\PY@it=\textit\def\PY@tc##1{\textcolor[rgb]{0.25,0.50,0.50}{##1}}}
\expandafter\def\csname PY@tok@gh\endcsname{\let\PY@bf=\textbf\def\PY@tc##1{\textcolor[rgb]{0.00,0.00,0.50}{##1}}}
\expandafter\def\csname PY@tok@kr\endcsname{\let\PY@bf=\textbf\def\PY@tc##1{\textcolor[rgb]{0.00,0.50,0.00}{##1}}}
\expandafter\def\csname PY@tok@go\endcsname{\def\PY@tc##1{\textcolor[rgb]{0.53,0.53,0.53}{##1}}}
\expandafter\def\csname PY@tok@gs\endcsname{\let\PY@bf=\textbf}
\expandafter\def\csname PY@tok@cm\endcsname{\let\PY@it=\textit\def\PY@tc##1{\textcolor[rgb]{0.25,0.50,0.50}{##1}}}
\expandafter\def\csname PY@tok@kp\endcsname{\def\PY@tc##1{\textcolor[rgb]{0.00,0.50,0.00}{##1}}}
\expandafter\def\csname PY@tok@ge\endcsname{\let\PY@it=\textit}
\expandafter\def\csname PY@tok@nn\endcsname{\let\PY@bf=\textbf\def\PY@tc##1{\textcolor[rgb]{0.00,0.00,1.00}{##1}}}
\expandafter\def\csname PY@tok@kd\endcsname{\let\PY@bf=\textbf\def\PY@tc##1{\textcolor[rgb]{0.00,0.50,0.00}{##1}}}
\expandafter\def\csname PY@tok@na\endcsname{\def\PY@tc##1{\textcolor[rgb]{0.49,0.56,0.16}{##1}}}
\expandafter\def\csname PY@tok@gr\endcsname{\def\PY@tc##1{\textcolor[rgb]{1.00,0.00,0.00}{##1}}}
\expandafter\def\csname PY@tok@o\endcsname{\def\PY@tc##1{\textcolor[rgb]{0.40,0.40,0.40}{##1}}}
\expandafter\def\csname PY@tok@m\endcsname{\def\PY@tc##1{\textcolor[rgb]{0.40,0.40,0.40}{##1}}}
\expandafter\def\csname PY@tok@gp\endcsname{\let\PY@bf=\textbf\def\PY@tc##1{\textcolor[rgb]{0.00,0.00,0.50}{##1}}}
\expandafter\def\csname PY@tok@mo\endcsname{\def\PY@tc##1{\textcolor[rgb]{0.40,0.40,0.40}{##1}}}
\expandafter\def\csname PY@tok@gd\endcsname{\def\PY@tc##1{\textcolor[rgb]{0.63,0.00,0.00}{##1}}}
\expandafter\def\csname PY@tok@k\endcsname{\let\PY@bf=\textbf\def\PY@tc##1{\textcolor[rgb]{0.00,0.50,0.00}{##1}}}
\expandafter\def\csname PY@tok@ne\endcsname{\let\PY@bf=\textbf\def\PY@tc##1{\textcolor[rgb]{0.82,0.25,0.23}{##1}}}
\expandafter\def\csname PY@tok@vc\endcsname{\def\PY@tc##1{\textcolor[rgb]{0.10,0.09,0.49}{##1}}}
\expandafter\def\csname PY@tok@bp\endcsname{\def\PY@tc##1{\textcolor[rgb]{0.00,0.50,0.00}{##1}}}
\expandafter\def\csname PY@tok@sc\endcsname{\def\PY@tc##1{\textcolor[rgb]{0.73,0.13,0.13}{##1}}}
\expandafter\def\csname PY@tok@ss\endcsname{\def\PY@tc##1{\textcolor[rgb]{0.10,0.09,0.49}{##1}}}
\expandafter\def\csname PY@tok@cp\endcsname{\def\PY@tc##1{\textcolor[rgb]{0.74,0.48,0.00}{##1}}}
\expandafter\def\csname PY@tok@nb\endcsname{\def\PY@tc##1{\textcolor[rgb]{0.00,0.50,0.00}{##1}}}
\expandafter\def\csname PY@tok@gt\endcsname{\def\PY@tc##1{\textcolor[rgb]{0.00,0.27,0.87}{##1}}}
\expandafter\def\csname PY@tok@mh\endcsname{\def\PY@tc##1{\textcolor[rgb]{0.40,0.40,0.40}{##1}}}
\expandafter\def\csname PY@tok@il\endcsname{\def\PY@tc##1{\textcolor[rgb]{0.40,0.40,0.40}{##1}}}
\expandafter\def\csname PY@tok@nl\endcsname{\def\PY@tc##1{\textcolor[rgb]{0.63,0.63,0.00}{##1}}}
\expandafter\def\csname PY@tok@nv\endcsname{\def\PY@tc##1{\textcolor[rgb]{0.10,0.09,0.49}{##1}}}
\expandafter\def\csname PY@tok@nd\endcsname{\def\PY@tc##1{\textcolor[rgb]{0.67,0.13,1.00}{##1}}}
\expandafter\def\csname PY@tok@w\endcsname{\def\PY@tc##1{\textcolor[rgb]{0.73,0.73,0.73}{##1}}}
\expandafter\def\csname PY@tok@nf\endcsname{\def\PY@tc##1{\textcolor[rgb]{0.00,0.00,1.00}{##1}}}
\expandafter\def\csname PY@tok@ow\endcsname{\let\PY@bf=\textbf\def\PY@tc##1{\textcolor[rgb]{0.67,0.13,1.00}{##1}}}
\expandafter\def\csname PY@tok@ni\endcsname{\let\PY@bf=\textbf\def\PY@tc##1{\textcolor[rgb]{0.60,0.60,0.60}{##1}}}
\expandafter\def\csname PY@tok@sd\endcsname{\let\PY@it=\textit\def\PY@tc##1{\textcolor[rgb]{0.73,0.13,0.13}{##1}}}
\expandafter\def\csname PY@tok@se\endcsname{\let\PY@bf=\textbf\def\PY@tc##1{\textcolor[rgb]{0.73,0.40,0.13}{##1}}}
\expandafter\def\csname PY@tok@mi\endcsname{\def\PY@tc##1{\textcolor[rgb]{0.40,0.40,0.40}{##1}}}
\expandafter\def\csname PY@tok@err\endcsname{\def\PY@bc##1{\setlength{\fboxsep}{0pt}\fcolorbox[rgb]{1.00,0.00,0.00}{1,1,1}{\strut ##1}}}
\expandafter\def\csname PY@tok@cs\endcsname{\let\PY@it=\textit\def\PY@tc##1{\textcolor[rgb]{0.25,0.50,0.50}{##1}}}
\expandafter\def\csname PY@tok@s2\endcsname{\def\PY@tc##1{\textcolor[rgb]{0.73,0.13,0.13}{##1}}}
\expandafter\def\csname PY@tok@kn\endcsname{\let\PY@bf=\textbf\def\PY@tc##1{\textcolor[rgb]{0.00,0.50,0.00}{##1}}}
\expandafter\def\csname PY@tok@sx\endcsname{\def\PY@tc##1{\textcolor[rgb]{0.00,0.50,0.00}{##1}}}
\expandafter\def\csname PY@tok@nc\endcsname{\let\PY@bf=\textbf\def\PY@tc##1{\textcolor[rgb]{0.00,0.00,1.00}{##1}}}
\expandafter\def\csname PY@tok@nt\endcsname{\let\PY@bf=\textbf\def\PY@tc##1{\textcolor[rgb]{0.00,0.50,0.00}{##1}}}
\expandafter\def\csname PY@tok@vi\endcsname{\def\PY@tc##1{\textcolor[rgb]{0.10,0.09,0.49}{##1}}}
\expandafter\def\csname PY@tok@gu\endcsname{\let\PY@bf=\textbf\def\PY@tc##1{\textcolor[rgb]{0.50,0.00,0.50}{##1}}}
\expandafter\def\csname PY@tok@no\endcsname{\def\PY@tc##1{\textcolor[rgb]{0.53,0.00,0.00}{##1}}}
\expandafter\def\csname PY@tok@sb\endcsname{\def\PY@tc##1{\textcolor[rgb]{0.73,0.13,0.13}{##1}}}
\expandafter\def\csname PY@tok@kc\endcsname{\let\PY@bf=\textbf\def\PY@tc##1{\textcolor[rgb]{0.00,0.50,0.00}{##1}}}

\def\PYZbs{\char`\\}
\def\PYZus{\char`\_}
\def\PYZob{\char`\{}
\def\PYZcb{\char`\}}
\def\PYZca{\char`\^}
\def\PYZam{\char`\&}
\def\PYZlt{\char`\<}
\def\PYZgt{\char`\>}
\def\PYZsh{\char`\#}
\def\PYZpc{\char`\%}
\def\PYZdl{\char`\$}
\def\PYZhy{\char`\-}
\def\PYZsq{\char`\'}
\def\PYZdq{\char`\"}
\def\PYZti{\char`\~}
% for compatibility with earlier versions
\def\PYZat{@}
\def\PYZlb{[}
\def\PYZrb{]}
\makeatother


    % Exact colors from NB
    \definecolor{incolor}{rgb}{0.0, 0.0, 0.5}
    \definecolor{outcolor}{rgb}{0.545, 0.0, 0.0}



    
    % Prevent overflowing lines due to hard-to-break entities
    \sloppy 
    % Setup hyperref package
    \hypersetup{
      breaklinks=true,  % so long urls are correctly broken across lines
      colorlinks=true,
      urlcolor=blue,
      linkcolor=darkorange,
      citecolor=darkgreen,
      }
    % Slightly bigger margins than the latex defaults
    
    \geometry{verbose,tmargin=1in,bmargin=1in,lmargin=1in,rmargin=1in}
    
    

    \begin{document}
    
    
    \maketitle
    
    

    
    \begin{Verbatim}[commandchars=\\\{\}]
{\color{incolor}In [{\color{incolor}}]:} \PY{k+kn}{from} \PY{n+nn}{IPython.display} \PY{k+kn}{import} \PY{n}{Image}
       \PY{k+kn}{from} \PY{n+nn}{IPython.display} \PY{k+kn}{import} \PY{n}{FileLink}\PY{p}{,} \PY{n}{FileLinks}
\end{Verbatim}

    \begin{Verbatim}[commandchars=\\\{\}]
{\color{incolor}In [{\color{incolor}}]:} 
\end{Verbatim}

    \begin{Verbatim}[commandchars=\\\{\}]
{\color{incolor}In [{\color{incolor}}]:} 
\end{Verbatim}

    \begin{Verbatim}[commandchars=\\\{\}]
{\color{incolor}In [{\color{incolor}}]:} 
\end{Verbatim}


    \section{Homework 1: Classical Planning}


    Robot Intelligence - Planning CS 4649/7649, Fall 2014\\Instructor:
Sungmoon Joo

9/29/14 \#\# Team 3 * Siddharth Choudhary * Varun Murali * Yosef Razin *
Ruffin White


    \subsection{\begin{enumerate}
\def\labelenumi{\arabic{enumi})}
\itemsep1pt\parskip0pt\parsep0pt
\item
  Towers of Hanoi
\end{enumerate}}


    A famous problem in classical planning is the Towers of Hanoi.
Apparently, some priests in Vietnam are required to stack enormous discs
from one tower to another by command of an ancient prophecy. Lets help
them out with modern automation. The discs must always be stacked in
order of increasing height. The goal is to move all the discs from the
first tower to the third. Experiment with at least two different
classical planners to solve this problem. Links to the domain are
provided on the course web page. The page also contains links to some
recommended planners. You are welcome and recommended to try other
planners as well.

    \begin{Verbatim}[commandchars=\\\{\}]
{\color{incolor}In [{\color{incolor}}]:} \PY{n}{Image}\PY{p}{(}\PY{l+s}{\PYZdq{}}\PY{l+s}{Figures/Towers of Hanoi.png}\PY{l+s}{\PYZdq{}}\PY{p}{)}
\end{Verbatim}

    We'll specify the relative path to each of the planners

    \begin{Verbatim}[commandchars=\\\{\}]
{\color{incolor}In [{\color{incolor}}]:} \PY{n}{planner\PYZus{}path} \PY{o}{=} \PY{l+s}{\PYZdq{}}\PY{l+s}{../../../Documents/Code/Planners/}\PY{l+s}{\PYZdq{}}
       \PY{n}{blackbox}  \PY{o}{=} \PY{n}{planner\PYZus{}path} \PY{o}{+} \PY{l+s}{\PYZdq{}}\PY{l+s}{Blackbox43LinuxBinary/blackbox}\PY{l+s}{\PYZdq{}}
       \PY{n}{satplan}   \PY{o}{=} \PY{n}{planner\PYZus{}path} \PY{o}{+} \PY{l+s}{\PYZdq{}}\PY{l+s}{SatPlan2006\PYZus{}LinuxBin/satplan}\PY{l+s}{\PYZdq{}}
       \PY{n}{vhpop}     \PY{o}{=} \PY{n}{planner\PYZus{}path} \PY{o}{+} \PY{l+s}{\PYZdq{}}\PY{l+s}{vhpop\PYZhy{}2.2.1/vhpop}\PY{l+s}{\PYZdq{}}
       \PY{n}{graphplan} \PY{o}{=} \PY{n}{planner\PYZus{}path} \PY{o}{+} \PY{l+s}{\PYZdq{}}\PY{l+s}{Graphplan/graphplan}\PY{l+s}{\PYZdq{}}
       \PY{n}{pyperplan} \PY{o}{=} \PY{n}{planner\PYZus{}path} \PY{o}{+} \PY{l+s}{\PYZdq{}}\PY{l+s}{pyperplan/src/pyperplan.py}\PY{l+s}{\PYZdq{}}
       \PY{c}{\PYZsh{} shop      = \PYZdq{}../../../Documents/Code/Planners/shop2\PYZhy{}2.9.0/\PYZdq{}}
\end{Verbatim}

    And then the relative path of the domain and problem .pddl files

    \begin{Verbatim}[commandchars=\\\{\}]
{\color{incolor}In [{\color{incolor}}]:} \PY{n}{tower\PYZus{}path} \PY{o}{=} \PY{l+s}{\PYZdq{}}\PY{l+s}{/home/tox/git/HW1\PYZus{}Team3/Code/Towers/}\PY{l+s}{\PYZdq{}}
       \PY{n}{domain}   \PY{o}{=} \PY{n}{tower\PYZus{}path} \PY{o}{+} \PY{l+s}{\PYZdq{}}\PY{l+s}{hanoi\PYZhy{}domain.pddl}\PY{l+s}{\PYZdq{}}
       \PY{n}{problem}  \PY{o}{=} \PY{n}{tower\PYZus{}path} \PY{o}{+} \PY{l+s}{\PYZdq{}}\PY{l+s}{hanoi2.pddl}\PY{l+s}{\PYZdq{}}
\end{Verbatim}

    Lets go ahead and try out the blackbox planner

    \begin{Verbatim}[commandchars=\\\{\}]
{\color{incolor}In [{\color{incolor}}]:} \PY{o}{\PYZpc{}\PYZpc{}}\PY{k}{bash} \PY{o}{\PYZhy{}}\PY{n}{s} \PY{l+s}{\PYZdq{}}\PY{l+s}{\PYZdl{}blackbox}\PY{l+s}{\PYZdq{}} \PY{l+s}{\PYZdq{}}\PY{l+s}{\PYZdl{}domain}\PY{l+s}{\PYZdq{}} \PY{l+s}{\PYZdq{}}\PY{l+s}{\PYZdl{}problem}\PY{l+s}{\PYZdq{}}
       \PY{err}{\PYZdl{}}\PY{l+m+mi}{1} \PY{o}{\PYZhy{}}\PY{n}{o} \PY{err}{\PYZdl{}}\PY{l+m+mi}{2} \PY{o}{\PYZhy{}}\PY{n}{f} \PY{err}{\PYZdl{}}\PY{l+m+mi}{3} \PY{o}{\PYZhy{}}\PY{n}{solver} \PY{n}{graphplan}
\end{Verbatim}

    Ok, so that worked out. Now lets try VHPOP.

    \begin{Verbatim}[commandchars=\\\{\}]
{\color{incolor}In [{\color{incolor}}]:} \PY{o}{\PYZpc{}\PYZpc{}}\PY{k}{bash} \PY{o}{\PYZhy{}}\PY{n}{s} \PY{l+s}{\PYZdq{}}\PY{l+s}{\PYZdl{}vhpop}\PY{l+s}{\PYZdq{}} \PY{l+s}{\PYZdq{}}\PY{l+s}{\PYZdl{}domain}\PY{l+s}{\PYZdq{}} \PY{l+s}{\PYZdq{}}\PY{l+s}{\PYZdl{}problem}\PY{l+s}{\PYZdq{}}
       \PY{err}{\PYZdl{}}\PY{l+m+mi}{1}  \PY{o}{\PYZhy{}}\PY{n}{f} \PY{n}{LCFR} \PY{o}{\PYZhy{}}\PY{n}{l} \PY{l+m+mi}{10000} \PY{o}{\PYZhy{}}\PY{n}{f} \PY{n}{MW} \PY{o}{\PYZhy{}}\PY{n}{l} \PY{n}{unlimited} \PY{err}{\PYZdl{}}\PY{l+m+mi}{2} \PY{err}{\PYZdl{}}\PY{l+m+mi}{3}
\end{Verbatim}


    \subsubsection{\begin{enumerate}
\def\labelenumi{\alph{enumi})}
\itemsep1pt\parskip0pt\parsep0pt
\item
  Explain the method by which each of the two planners finds a solution.
\end{enumerate}}


    The first planner applied, Blackbox, is a planning system that compbines
satplan and graphplan. Basicly, it parses the PDDL files specified with
STRIPS notation into a Boolean, or propositional, satisfiability problem
and then applies several diffrent types of satisfiability engines. The
name blackbox refers to that the plan generator and the SAT solver know
nothing about eachother, thus premitting a flexable sytem to try out
diffrent engines to use. The perticular solver used here is just
graphplan with defoult perameters.

The second planner applied, VHPOP is a partial order causal link (POCL)
planner loosely based on UCPOP from University of Washington. Written by
Håkan L. S. Younes, VHPOP gained reconition durring 3rd International
Planning Competition (2002) as Best Newcomer and thus reviving the study
partial order planning.


    \subsubsection{\begin{enumerate}
\def\labelenumi{\alph{enumi})}
\setcounter{enumi}{1}
\itemsep1pt\parskip0pt\parsep0pt
\item
  Which planner was fastest?
\end{enumerate}}


    \begin{Verbatim}[commandchars=\\\{\}]
{\color{incolor}In [{\color{incolor}}]:} 
\end{Verbatim}


    \subsubsection{\begin{enumerate}
\def\labelenumi{\alph{enumi})}
\setcounter{enumi}{2}
\itemsep1pt\parskip0pt\parsep0pt
\item
  Explain why the winning planner might be more effective on this
  problem.
\end{enumerate}}


    \begin{Verbatim}[commandchars=\\\{\}]
{\color{incolor}In [{\color{incolor}}]:} 
\end{Verbatim}


    \subsection{\begin{enumerate}
\def\labelenumi{\arabic{enumi})}
\setcounter{enumi}{1}
\itemsep1pt\parskip0pt\parsep0pt
\item
  Sokoban PDDL
\end{enumerate}}


    During the times of Pong, Pac-Man and Tetris, Hiroyuki Imabayashi
created an complex game that tested the human abilities of planning:
Sokoban. Many folks are still addicted to solving Sokoban puzzles and
you can join them by playing any of the versions freely distributed on
the web. The goal is for the human, or robot, to push all the boxes into
the desired locations. The robot can move horizontally and vertically
and push one box at a time.

    \begin{Verbatim}[commandchars=\\\{\}]
{\color{incolor}In [{\color{incolor}}]:} \PY{n}{Image}\PY{p}{(}\PY{l+s}{\PYZdq{}}\PY{l+s}{Figures/Sokoban 1.png}\PY{l+s}{\PYZdq{}}\PY{p}{)}
\end{Verbatim}

    Describe the Sokoban domain in PDDL. For each of the problems in Figure
2, define the problem in PDDL. You can either use the target lettering
given in the picture or let the planner move any box to any target
square. For the challenge problem, any box in any location is a
solution. In the challenge, PDDL should NOT inform the robot which box
should go to which location. In addition you may also try other problems
you invent or find on the web. How well do your two planners perform on
these problems? If no planner seems to be solving it, perhaps you should
consider a different method for defining your problems.


    \subsubsection{\begin{enumerate}
\def\labelenumi{\alph{enumi})}
\itemsep1pt\parskip0pt\parsep0pt
\item
  Show successful plans from at least one planner on the three Sokoban
  problems in Figure 2(1-3). The challenge problem is optional.
\end{enumerate}}


    \begin{Verbatim}[commandchars=\\\{\}]
{\color{incolor}In [{\color{incolor}}]:} \PY{o}{\PYZpc{}}\PY{k}{run} \PY{l+s}{\PYZsq{}}\PY{l+s}{Scratch Book.ipynb}\PY{l+s}{\PYZsq{}}
\end{Verbatim}

    \begin{Verbatim}[commandchars=\\\{\}]
{\color{incolor}In [{\color{incolor}}]:} \PY{n}{world\PYZus{}path} \PY{o}{=} \PY{l+s}{\PYZdq{}}\PY{l+s}{/home/tox/git/HW1\PYZus{}Team3/Code/Worlds/}\PY{l+s}{\PYZdq{}}
       
       \PY{n}{domain}   \PY{o}{=} \PY{n}{world\PYZus{}path} \PY{o}{+} \PY{l+s}{\PYZdq{}}\PY{l+s}{sokoban\PYZus{}domain.pddl}\PY{l+s}{\PYZdq{}}
       \PY{n}{world}    \PY{o}{=} \PY{n}{world\PYZus{}path} \PY{o}{+} \PY{l+s}{\PYZdq{}}\PY{l+s}{world1.txt}\PY{l+s}{\PYZdq{}}
       \PY{n}{problem}  \PY{o}{=} \PY{n}{world\PYZus{}path} \PY{o}{+} \PY{l+s}{\PYZdq{}}\PY{l+s}{world1\PYZus{}proble.pddl}\PY{l+s}{\PYZdq{}}
       \PY{n}{solution} \PY{o}{=} \PY{n}{world\PYZus{}path} \PY{o}{+} \PY{l+s}{\PYZdq{}}\PY{l+s}{world1\PYZus{}proble.pddl.soln}\PY{l+s}{\PYZdq{}}
       \PY{n}{graph}    \PY{o}{=} \PY{n}{world\PYZus{}path} \PY{o}{+} \PY{l+s}{\PYZdq{}}\PY{l+s}{world1\PYZus{}proble.png}\PY{l+s}{\PYZdq{}}
       
       \PY{n}{sokoban} \PY{o}{=} \PY{n}{Sokoban}\PY{p}{(}\PY{n}{world}\PY{p}{,} \PY{n}{labeled\PYZus{}boxes} \PY{o}{=} \PY{n+nb+bp}{False}\PY{p}{)}
       \PY{n}{sokoban}\PY{o}{.}\PY{n}{writeProblem}\PY{p}{(}\PY{n}{problem}\PY{p}{)}
\end{Verbatim}

    \begin{Verbatim}[commandchars=\\\{\}]
{\color{incolor}In [{\color{incolor}}]:} \PY{o}{\PYZpc{}\PYZpc{}}\PY{k}{bash} \PY{o}{\PYZhy{}}\PY{n}{s} \PY{l+s}{\PYZdq{}}\PY{l+s}{cat}\PY{l+s}{\PYZdq{}} \PY{l+s}{\PYZdq{}}\PY{l+s}{\PYZdl{}domain}\PY{l+s}{\PYZdq{}}
       \PY{err}{\PYZdl{}}\PY{l+m+mi}{1} \PY{err}{\PYZdl{}}\PY{l+m+mi}{2}
\end{Verbatim}

    \begin{Verbatim}[commandchars=\\\{\}]
{\color{incolor}In [{\color{incolor}}]:} \PY{o}{\PYZpc{}\PYZpc{}}\PY{k}{bash} \PY{o}{\PYZhy{}}\PY{n}{s} \PY{l+s}{\PYZdq{}}\PY{l+s}{cat}\PY{l+s}{\PYZdq{}} \PY{l+s}{\PYZdq{}}\PY{l+s}{\PYZdl{}world}\PY{l+s}{\PYZdq{}}
       \PY{err}{\PYZdl{}}\PY{l+m+mi}{1} \PY{err}{\PYZdl{}}\PY{l+m+mi}{2}
\end{Verbatim}

    \begin{Verbatim}[commandchars=\\\{\}]
{\color{incolor}In [{\color{incolor}}]:} \PY{o}{\PYZpc{}\PYZpc{}}\PY{k}{bash} \PY{o}{\PYZhy{}}\PY{n}{s} \PY{l+s}{\PYZdq{}}\PY{l+s}{cat}\PY{l+s}{\PYZdq{}} \PY{l+s}{\PYZdq{}}\PY{l+s}{\PYZdl{}problem}\PY{l+s}{\PYZdq{}}
       \PY{err}{\PYZdl{}}\PY{l+m+mi}{1} \PY{err}{\PYZdl{}}\PY{l+m+mi}{2}
\end{Verbatim}

    \begin{Verbatim}[commandchars=\\\{\}]
{\color{incolor}In [{\color{incolor}}]:} \PY{o}{\PYZpc{}\PYZpc{}}\PY{k}{bash} \PY{o}{\PYZhy{}}\PY{n}{s} \PY{l+s}{\PYZdq{}}\PY{l+s}{\PYZdl{}blackbox}\PY{l+s}{\PYZdq{}} \PY{l+s}{\PYZdq{}}\PY{l+s}{\PYZdl{}domain}\PY{l+s}{\PYZdq{}} \PY{l+s}{\PYZdq{}}\PY{l+s}{\PYZdl{}problem}\PY{l+s}{\PYZdq{}}
       \PY{err}{\PYZdl{}}\PY{l+m+mi}{1} \PY{o}{\PYZhy{}}\PY{n}{o} \PY{err}{\PYZdl{}}\PY{l+m+mi}{2} \PY{o}{\PYZhy{}}\PY{n}{f} \PY{err}{\PYZdl{}}\PY{l+m+mi}{3}
\end{Verbatim}

    \begin{Verbatim}[commandchars=\\\{\}]
{\color{incolor}In [{\color{incolor}}]:} \PY{o}{\PYZpc{}\PYZpc{}}\PY{k}{bash} \PY{o}{\PYZhy{}}\PY{n}{s} \PY{l+s}{\PYZdq{}}\PY{l+s}{\PYZdl{}vhpop}\PY{l+s}{\PYZdq{}} \PY{l+s}{\PYZdq{}}\PY{l+s}{\PYZdl{}domain}\PY{l+s}{\PYZdq{}} \PY{l+s}{\PYZdq{}}\PY{l+s}{\PYZdl{}problem}\PY{l+s}{\PYZdq{}}
       \PY{err}{\PYZdl{}}\PY{l+m+mi}{1}  \PY{o}{\PYZhy{}}\PY{n}{f} \PY{n}{LCFR} \PY{o}{\PYZhy{}}\PY{n}{l} \PY{l+m+mi}{10000} \PY{o}{\PYZhy{}}\PY{n}{f} \PY{n}{MW} \PY{o}{\PYZhy{}}\PY{n}{l} \PY{n}{unlimited} \PY{err}{\PYZdl{}}\PY{l+m+mi}{2} \PY{err}{\PYZdl{}}\PY{l+m+mi}{3}
\end{Verbatim}

    Well, VHPOP never seamed to be able to finish this example under 15 min,
so lets try somthingelse.

    \begin{Verbatim}[commandchars=\\\{\}]
{\color{incolor}In [{\color{incolor}}]:} \PY{o}{\PYZpc{}\PYZpc{}}\PY{k}{bash} \PY{o}{\PYZhy{}}\PY{n}{s} \PY{l+s}{\PYZdq{}}\PY{l+s}{\PYZdl{}pyperplan}\PY{l+s}{\PYZdq{}} \PY{l+s}{\PYZdq{}}\PY{l+s}{\PYZdl{}domain}\PY{l+s}{\PYZdq{}} \PY{l+s}{\PYZdq{}}\PY{l+s}{\PYZdl{}problem}\PY{l+s}{\PYZdq{}}
       \PY{err}{\PYZdl{}}\PY{l+m+mi}{1} \PY{err}{\PYZdl{}}\PY{l+m+mi}{2} \PY{err}{\PYZdl{}}\PY{l+m+mi}{3} \PY{err}{\PYZdl{}}\PY{l+m+mi}{4} \PY{o}{\PYZhy{}}\PY{o}{\PYZhy{}}\PY{n}{plugins} \PY{n}{visualizer} \PY{o}{\PYZhy{}}\PY{n}{s} \PY{n}{bfs}
\end{Verbatim}

    \begin{Verbatim}[commandchars=\\\{\}]
{\color{incolor}In [{\color{incolor}}]:} \PY{o}{\PYZpc{}\PYZpc{}}\PY{k}{bash} \PY{o}{\PYZhy{}}\PY{n}{s} \PY{l+s}{\PYZdq{}}\PY{l+s}{cat}\PY{l+s}{\PYZdq{}} \PY{l+s}{\PYZdq{}}\PY{l+s}{\PYZdl{}solution}\PY{l+s}{\PYZdq{}}
       \PY{err}{\PYZdl{}}\PY{l+m+mi}{1} \PY{err}{\PYZdl{}}\PY{l+m+mi}{2}
\end{Verbatim}

    \begin{Verbatim}[commandchars=\\\{\}]
{\color{incolor}In [{\color{incolor}}]:} \PY{n}{Image}\PY{p}{(}\PY{l+s}{\PYZdq{}}\PY{l+s}{Worlds\PYZhy{}bfs\PYZhy{}hff\PYZhy{}world1\PYZus{}proble.png}\PY{l+s}{\PYZdq{}}\PY{p}{)}
\end{Verbatim}

    \begin{Verbatim}[commandchars=\\\{\}]
{\color{incolor}In [{\color{incolor}}]:} \PY{n}{domain}   \PY{o}{=} \PY{n}{world\PYZus{}path} \PY{o}{+} \PY{l+s}{\PYZdq{}}\PY{l+s}{sokoban\PYZus{}domain.pddl}\PY{l+s}{\PYZdq{}}
       \PY{n}{world}    \PY{o}{=} \PY{n}{world\PYZus{}path} \PY{o}{+} \PY{l+s}{\PYZdq{}}\PY{l+s}{world2.txt}\PY{l+s}{\PYZdq{}}
       \PY{n}{problem}  \PY{o}{=} \PY{n}{world\PYZus{}path} \PY{o}{+} \PY{l+s}{\PYZdq{}}\PY{l+s}{world2\PYZus{}proble.pddl}\PY{l+s}{\PYZdq{}}
       \PY{n}{solution} \PY{o}{=} \PY{n}{world\PYZus{}path} \PY{o}{+} \PY{l+s}{\PYZdq{}}\PY{l+s}{world2\PYZus{}proble.pddl.soln}\PY{l+s}{\PYZdq{}}
       
       \PY{n}{sokoban} \PY{o}{=} \PY{n}{Sokoban}\PY{p}{(}\PY{n}{world}\PY{p}{,} \PY{n}{labeled\PYZus{}boxes} \PY{o}{=} \PY{n+nb+bp}{False}\PY{p}{)}
       \PY{n}{sokoban}\PY{o}{.}\PY{n}{writeProblem}\PY{p}{(}\PY{n}{problem}\PY{p}{)}
\end{Verbatim}

    \begin{Verbatim}[commandchars=\\\{\}]
{\color{incolor}In [{\color{incolor}}]:} \PY{o}{\PYZpc{}\PYZpc{}}\PY{k}{bash} \PY{o}{\PYZhy{}}\PY{n}{s} \PY{l+s}{\PYZdq{}}\PY{l+s}{\PYZdl{}pyperplan}\PY{l+s}{\PYZdq{}} \PY{l+s}{\PYZdq{}}\PY{l+s}{\PYZdl{}domain}\PY{l+s}{\PYZdq{}} \PY{l+s}{\PYZdq{}}\PY{l+s}{\PYZdl{}problem}\PY{l+s}{\PYZdq{}}
       \PY{err}{\PYZdl{}}\PY{l+m+mi}{1} \PY{err}{\PYZdl{}}\PY{l+m+mi}{2} \PY{err}{\PYZdl{}}\PY{l+m+mi}{3} \PY{err}{\PYZdl{}}\PY{l+m+mi}{4} \PY{o}{\PYZhy{}}\PY{o}{\PYZhy{}}\PY{n}{plugins} \PY{n}{visualizer}
\end{Verbatim}

    \begin{Verbatim}[commandchars=\\\{\}]
{\color{incolor}In [{\color{incolor}}]:} \PY{o}{\PYZpc{}\PYZpc{}}\PY{k}{bash} \PY{o}{\PYZhy{}}\PY{n}{s} \PY{l+s}{\PYZdq{}}\PY{l+s}{cat}\PY{l+s}{\PYZdq{}} \PY{l+s}{\PYZdq{}}\PY{l+s}{\PYZdl{}solution}\PY{l+s}{\PYZdq{}}
       \PY{err}{\PYZdl{}}\PY{l+m+mi}{1} \PY{err}{\PYZdl{}}\PY{l+m+mi}{2}
\end{Verbatim}

    \begin{Verbatim}[commandchars=\\\{\}]
{\color{incolor}In [{\color{incolor}}]:} \PY{n}{Image}\PY{p}{(}\PY{l+s}{\PYZdq{}}\PY{l+s}{Worlds\PYZhy{}bfs\PYZhy{}hff\PYZhy{}world2\PYZus{}proble.png}\PY{l+s}{\PYZdq{}}\PY{p}{)}
\end{Verbatim}

    \begin{Verbatim}[commandchars=\\\{\}]
{\color{incolor}In [{\color{incolor}}]:} \PY{o}{\PYZpc{}\PYZpc{}}\PY{k}{bash} \PY{o}{\PYZhy{}}\PY{n}{s} \PY{l+s}{\PYZdq{}}\PY{l+s}{\PYZdl{}pyperplan}\PY{l+s}{\PYZdq{}} \PY{l+s}{\PYZdq{}}\PY{l+s}{\PYZdl{}domain}\PY{l+s}{\PYZdq{}} \PY{l+s}{\PYZdq{}}\PY{l+s}{\PYZdl{}problem}\PY{l+s}{\PYZdq{}}
       \PY{err}{\PYZdl{}}\PY{l+m+mi}{1} \PY{err}{\PYZdl{}}\PY{l+m+mi}{2} \PY{err}{\PYZdl{}}\PY{l+m+mi}{3} \PY{err}{\PYZdl{}}\PY{l+m+mi}{4} \PY{o}{\PYZhy{}}\PY{o}{\PYZhy{}}\PY{n}{plugins} \PY{n}{visualizer} \PY{o}{\PYZhy{}}\PY{n}{H} \PY{n}{hff} \PY{o}{\PYZhy{}}\PY{n}{s} \PY{n}{gbf}
\end{Verbatim}

    \begin{Verbatim}[commandchars=\\\{\}]
{\color{incolor}In [{\color{incolor}}]:} \PY{n}{Image}\PY{p}{(}\PY{l+s}{\PYZdq{}}\PY{l+s}{Worlds\PYZhy{}gbf\PYZhy{}hff\PYZhy{}world2\PYZus{}proble.png}\PY{l+s}{\PYZdq{}}\PY{p}{)}
\end{Verbatim}

    \begin{Verbatim}[commandchars=\\\{\}]
{\color{incolor}In [{\color{incolor}}]:} \PY{n}{domain}   \PY{o}{=} \PY{n}{world\PYZus{}path} \PY{o}{+} \PY{l+s}{\PYZdq{}}\PY{l+s}{sokoban\PYZus{}domain.pddl}\PY{l+s}{\PYZdq{}}
       \PY{n}{world}    \PY{o}{=} \PY{n}{world\PYZus{}path} \PY{o}{+} \PY{l+s}{\PYZdq{}}\PY{l+s}{world3.txt}\PY{l+s}{\PYZdq{}}
       \PY{n}{problem}  \PY{o}{=} \PY{n}{world\PYZus{}path} \PY{o}{+} \PY{l+s}{\PYZdq{}}\PY{l+s}{world3\PYZus{}proble.pddl}\PY{l+s}{\PYZdq{}}
       \PY{n}{solution} \PY{o}{=} \PY{n}{world\PYZus{}path} \PY{o}{+} \PY{l+s}{\PYZdq{}}\PY{l+s}{world3\PYZus{}proble.pddl.soln}\PY{l+s}{\PYZdq{}}
       
       \PY{n}{sokoban} \PY{o}{=} \PY{n}{Sokoban}\PY{p}{(}\PY{n}{world}\PY{p}{,} \PY{n}{labeled\PYZus{}boxes} \PY{o}{=} \PY{n+nb+bp}{False}\PY{p}{)}
       \PY{n}{sokoban}\PY{o}{.}\PY{n}{writeProblem}\PY{p}{(}\PY{n}{problem}\PY{p}{)}
\end{Verbatim}

    \begin{Verbatim}[commandchars=\\\{\}]
{\color{incolor}In [{\color{incolor}}]:} \PY{o}{\PYZpc{}\PYZpc{}}\PY{k}{bash} \PY{o}{\PYZhy{}}\PY{n}{s} \PY{l+s}{\PYZdq{}}\PY{l+s}{\PYZdl{}pyperplan}\PY{l+s}{\PYZdq{}} \PY{l+s}{\PYZdq{}}\PY{l+s}{\PYZdl{}domain}\PY{l+s}{\PYZdq{}} \PY{l+s}{\PYZdq{}}\PY{l+s}{\PYZdl{}problem}\PY{l+s}{\PYZdq{}}
       \PY{err}{\PYZdl{}}\PY{l+m+mi}{1} \PY{err}{\PYZdl{}}\PY{l+m+mi}{2} \PY{err}{\PYZdl{}}\PY{l+m+mi}{3} \PY{err}{\PYZdl{}}\PY{l+m+mi}{4} \PY{o}{\PYZhy{}}\PY{o}{\PYZhy{}}\PY{n}{plugins} \PY{n}{visualizer}
\end{Verbatim}

    \begin{Verbatim}[commandchars=\\\{\}]
{\color{incolor}In [{\color{incolor}}]:} \PY{o}{\PYZpc{}\PYZpc{}}\PY{k}{bash} \PY{o}{\PYZhy{}}\PY{n}{s} \PY{l+s}{\PYZdq{}}\PY{l+s}{cat}\PY{l+s}{\PYZdq{}} \PY{l+s}{\PYZdq{}}\PY{l+s}{\PYZdl{}solution}\PY{l+s}{\PYZdq{}}
       \PY{err}{\PYZdl{}}\PY{l+m+mi}{1} \PY{err}{\PYZdl{}}\PY{l+m+mi}{2}
\end{Verbatim}

    \begin{Verbatim}[commandchars=\\\{\}]
{\color{incolor}In [{\color{incolor}}]:} \PY{n}{Image}\PY{p}{(}\PY{l+s}{\PYZdq{}}\PY{l+s}{Worlds\PYZhy{}bfs\PYZhy{}hff\PYZhy{}world3\PYZus{}proble.png}\PY{l+s}{\PYZdq{}}\PY{p}{)}
\end{Verbatim}

    \begin{Verbatim}[commandchars=\\\{\}]
{\color{incolor}In [{\color{incolor}}]:} \PY{o}{\PYZpc{}\PYZpc{}}\PY{k}{bash} \PY{o}{\PYZhy{}}\PY{n}{s} \PY{l+s}{\PYZdq{}}\PY{l+s}{\PYZdl{}pyperplan}\PY{l+s}{\PYZdq{}} \PY{l+s}{\PYZdq{}}\PY{l+s}{\PYZdl{}domain}\PY{l+s}{\PYZdq{}} \PY{l+s}{\PYZdq{}}\PY{l+s}{\PYZdl{}problem}\PY{l+s}{\PYZdq{}}
       \PY{err}{\PYZdl{}}\PY{l+m+mi}{1} \PY{err}{\PYZdl{}}\PY{l+m+mi}{2} \PY{err}{\PYZdl{}}\PY{l+m+mi}{3} \PY{err}{\PYZdl{}}\PY{l+m+mi}{4} \PY{o}{\PYZhy{}}\PY{o}{\PYZhy{}}\PY{n}{plugins} \PY{n}{visualizer} \PY{o}{\PYZhy{}}\PY{n}{H} \PY{n}{hff} \PY{o}{\PYZhy{}}\PY{n}{s} \PY{n}{gbf}
\end{Verbatim}

    \begin{Verbatim}[commandchars=\\\{\}]
{\color{incolor}In [{\color{incolor}}]:} \PY{n}{Image}\PY{p}{(}\PY{l+s}{\PYZdq{}}\PY{l+s}{Worlds\PYZhy{}gbf\PYZhy{}hff\PYZhy{}world3\PYZus{}proble.png}\PY{l+s}{\PYZdq{}}\PY{p}{)}
\end{Verbatim}

    \begin{Verbatim}[commandchars=\\\{\}]
{\color{incolor}In [{\color{incolor}}]:} \PY{n}{world}    \PY{o}{=} \PY{l+s}{\PYZdq{}}\PY{l+s}{Worlds/world4.txt}\PY{l+s}{\PYZdq{}}
       \PY{n}{problem}  \PY{o}{=} \PY{l+s}{\PYZdq{}}\PY{l+s}{Worlds/world4\PYZus{}proble.pddl}\PY{l+s}{\PYZdq{}}
       \PY{n}{solution} \PY{o}{=} \PY{l+s}{\PYZdq{}}\PY{l+s}{Worlds/world4\PYZus{}proble.pddl.soln}\PY{l+s}{\PYZdq{}}
       \PY{c}{\PYZsh{} sokoban = Sokoban(world, labeled\PYZus{}boxes = False)}
       \PY{c}{\PYZsh{} sokoban.writeProblem(problem)}
\end{Verbatim}

    \begin{Verbatim}[commandchars=\\\{\}]
{\color{incolor}In [{\color{incolor}}]:} \PY{o}{\PYZpc{}\PYZpc{}}\PY{k}{bash} \PY{o}{\PYZhy{}}\PY{n}{s} \PY{l+s}{\PYZdq{}}\PY{l+s}{cat}\PY{l+s}{\PYZdq{}} \PY{l+s}{\PYZdq{}}\PY{l+s}{\PYZdl{}solution}\PY{l+s}{\PYZdq{}}
       \PY{err}{\PYZdl{}}\PY{l+m+mi}{1} \PY{err}{\PYZdl{}}\PY{l+m+mi}{2}
\end{Verbatim}

    \begin{Verbatim}[commandchars=\\\{\}]
{\color{incolor}In [{\color{incolor}}]:} 
\end{Verbatim}

    \begin{Verbatim}[commandchars=\\\{\}]
{\color{incolor}In [{\color{incolor}}]:} 
\end{Verbatim}


    \subsubsection{\begin{enumerate}
\def\labelenumi{\alph{enumi})}
\setcounter{enumi}{1}
\itemsep1pt\parskip0pt\parsep0pt
\item
  Compare the performance of two planners on this domain. Which one
  works better? Does this make sense, why?
\end{enumerate}}


    \begin{Verbatim}[commandchars=\\\{\}]
{\color{incolor}In [{\color{incolor}}]:} 
\end{Verbatim}


    \subsubsection{\begin{enumerate}
\def\labelenumi{\alph{enumi})}
\setcounter{enumi}{2}
\itemsep1pt\parskip0pt\parsep0pt
\item
  Clearly PDDL was not intended for this sort of application. Discuss
  the challenges in expressing geometric constraints in semantic
  planning.
\end{enumerate}}


    \begin{Verbatim}[commandchars=\\\{\}]
{\color{incolor}In [{\color{incolor}}]:} 
\end{Verbatim}


    \subsubsection{\begin{enumerate}
\def\labelenumi{\alph{enumi})}
\setcounter{enumi}{3}
\itemsep1pt\parskip0pt\parsep0pt
\item
  In many cases, geometric and dynamic planning are insufficient to
  describe a domain. Give an example of a problem that is best suited
  for sematic (classical) planning. Explain why a semantic
  representation would be desirable.
\end{enumerate}}


    \begin{Verbatim}[commandchars=\\\{\}]
{\color{incolor}In [{\color{incolor}}]:} 
\end{Verbatim}


    % Add a bibliography block to the postdoc
    
    
    
    \end{document}
